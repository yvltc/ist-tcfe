\section{Theoretical Analysis}
\label{sec:analysis}
The RC circuit presented in Figure \ref{fig:t2circuit} has eight nodes, eleven branches, four elementary meshes and thirteen loops. It is going to be attributed zero potential to one of the nodes, which we will call  \textit{reference} or \textit{ground node}.
Consequently, we will need seven node equations for each usage of the nodal method to get the voltages in the other nodes.

\subsection{Simulating the operating point for $t<0$}\label{sec2.1}

The voltage function in the voltage source is given by $v_s=V_s u(-t)+sin(2\pi ft)u(t)$, where $V_s$ is a given constant. Considering $t<0$ for the substitution we obtain $v_s=V_s u(-t)$. Remembering that $u$ is a function defined by parts that assumes the value of $0$ for $t<0$ and $1$ for $t\geq 0$, we get $v_s(t)=V_s$. This means that the voltage in the voltage source is constant. 
Given the relation 
\begin{equation} \label{eq1}
    i(t)=C \frac{d}{dt}v(t)
\end{equation}
we easily conclude that the current $I_C$ is always zero. 
Using the node method, we compute the voltages in each node and the currents flowing through each branch for $t<0$, solving the system of equations \ref{eq2}. The results are presented below in table \ref{tab2}.

\begin{equation} \label{eq2}
\begin{split}
-\frac{1}{R_1}V_1+\left(\frac{1}{R_1}+\frac{1}{R_2}+\frac{1}{R_3}\right)V_2-\frac{1}{R_2}V_3-\frac{1}{R_3}V_5&=0 \\
\left(K_b+\frac{1}{R_2}\right)V_2-\frac{1}{R_2}V_3-K_b V_5&=0 \\
-\frac{1}{R_3}V_2+\left(\frac{1}{R_3}+\frac{1}{R_4}+\frac{1}{R_5}\right)V_5-\frac{1}{R_5}V_6-\frac{1}{R_7}V_7+\frac{1}{R_7}V_8&=0 \\
K_b V_2-\left(K_b+\frac{1}{R_5}\right)V_5+\frac{1}{R_5}V_6&=0 \\
\left(\frac{1}{R_6}+\frac{1}{R_7}\right)V_7-\frac{1}{R_7}V_8&=0 \\
V_1&=V_s \\ 
V_5-\frac{K_d}{R_6}V_7-V_8&=0 \\   
\end{split}
\end{equation}

\begin{table}[h]                             

\centering                                  % para centr. tabela
                        
\def\arraystretch{1.2}                       % esp. entre linhas
\begin{tabular}{c|c}                    % lr = 2 col (esq/dir)
\hline                                  % linha topo

\textbf{Name}  & \textbf{Value [A or V]}\\     

\hline                                % linha
\input{Nodal1_tab}
\hline                                % linha final
\end{tabular}   

\caption{Nodal analysis results. All currents \textit{I} expressed in Ampère; all voltages \textit{V} expressed in Volt.}
\label{tab2}   
\end{table}                             
\FloatBarrier

\subsection{Simulating the operating point for $v_s(0)=0$}\label{sec2.2}

In this part of the theoretical analysis, it is suggested that we turn the voltage source off and substitute the capacitor by a voltage source $V_x = V_6-V_8$, where $V_6$ and $V_8$ assume the values obtained in the previous procedure, noted in table \ref{tab3}. The main use of this is to calculate the equivalent resistance of the entire RC circuit $R_{eq}$ as well as the time constant $\tau$.

\begin{equation} \label{eq3}
\begin{split}
-\frac{1}{R_1}V_1+\left(\frac{1}{R_1}+\frac{1}{R_2}+\frac{1}{R_3}\right)V_2-\frac{1}{R_2}V_3-\frac{1}{R_3}V_5&=0\\
\left(K_b+\frac{1}{R_2}\right)V_2-\frac{1}{R_2}V_3-K_b V_5&=0\\
\frac{1}{R_1}V_1-\frac{1}{R_1}V_2-\frac{1}{R_4}V_5-\frac{1}{R_6}V_7&=0\\
V_6-V_8&=V_x\\
\left(\frac{1}{R_6}+\frac{1}{R_7}\right)V_7-\frac{1}{R_7}V_8&=0\\
V_1&=V_s \\
V_5-\frac{K_d}{R_6}V_7-V_8&=0
\end{split}
\end{equation}

As expected, the system of equations \ref{eq3} used in this section only varies from the one in \ref{eq2} by two equations, the third and the forth, which would require the value of $I_x$. Note that, since the voltage source is turned off, $V_s$ = 0.
$I_x$ can be determined after computing the node voltages, using the relation
\begin{equation}
    I_x = K_b(V_2 - V_5) + \frac{V_6 - V_5}{R_5}
\end{equation}
Then, as per the suggestion, we calculate $R_{eq}$ = $\frac{V_x}{I_x}$. With the equivalent resistor determined, the time constant of the RC loop, $\tau$ = $R_{eq}$ $C$, is obtained.

The results of this analysis are shown in the table below, including all node voltages, branch currents, the equivalent resistor and the time constant:

\begin{table}[h]                             

\centering                                  % para centr. tabela
                        
\def\arraystretch{1.2}                       % esp. entre linhas


\begin{tabular}{c|c}                    % lr = 2 col (esq/dir)
\hline                                  % linha topo

\textbf{Name}  & \textbf{Value [A, V, $\Omega$ or s]}\\     

\hline                                % linha
\input{Nodal2_tab}
\hline                                % linha final
\end{tabular}   


\caption{Nodal analysis results. All currents \textit{I} expressed in Ampère; all voltages \textit{V} expressed in Volt; the resistance expressed in Ohm and $\tau$ in seconds }
\label{tab3}   
\end{table}                             
\FloatBarrier





\subsection{Natural Solution $v_{6n}(t)$}\label{sec2.3}

Considering the time interval [0,20]ms it is possible to compute the behaviour of the voltage in node 6. For that was used the capacitor voltage $V_x$ for $t<0$, which served as the initial condition. We know from the theory classes that the natural solution of an RC loop is a negative exponential:
\begin{equation}
    v_n(t) = Ae^{-\frac{t}{\tau}}
\end{equation}

The time constant was obtained in section \ref{sec2.2}. As for the constant $A$, its value can be determined from the initial condition $v_{6n}(0)$ = $V_x$. That leaves us with the natural solution for the voltage in node 6:
\begin{equation}
    v_{6n}(t) = V_x e^{-\frac{t}{\tau}}
\end{equation}

The plot of this function is shown in figure \ref{fig2}.

\begin{figure}[!htp] \centering
\includegraphics[width=0.6\textwidth]{natural.eps}
\caption{Natural solution $v_{6n}(t)$ in the interval [0,20]ms}
\label{fig2}
\end{figure}
\FloatBarrier

\subsection{Forced Solution $v_{6f}(t)$}\label{sec2.4}

Considering the same interval [0,20]ms and $f=1 KHz$, we used the node method once again to determine the forced solution, this time substituting the capacitance $C$ with its impedance $Z_C$. It is important to note that for a capacitor this change relies on C $\mapsto \frac{1}{j \omega C}$. Knowing the frequency in hertz, we easily compute $\omega=2 \pi f$.

The equations used in this section were similar to the equations shown before. Only a single equation was changed comparing to the system in section \ref{sec2.2}, the equation of node 6, resulting in the following system of equations in complex variables:
\begin{equation} \label{eq4}
\begin{split}
-\frac{1}{R_1}V_1+\left(\frac{1}{R_1}+\frac{1}{R_2}+\frac{1}{R_3}\right)V_2-\frac{1}{R_2}V_3-\frac{1}{R_3}V_5&=0\\
\left(K_b+\frac{1}{R_2}\right)V_2-\frac{1}{R_2}V_3-K_b V_5&=0\\
\frac{1}{R_1}V_1-\frac{1}{R_1}V_2-\frac{1}{R_4}V_5-\frac{1}{R_6}V_7&=0\\
K_bV_2 - (\frac{1}{R_5}+K_b)V_5 + (\frac{1}{R_5}+Z_C)V_6 - \frac{1}{Z_C}&=0\\
\left(\frac{1}{R_6}+\frac{1}{R_7}\right)V_7-\frac{1}{R_7}V_8&=0\\
V_1&=v_s \\
V_5-\frac{K_d}{R_6}V_7-V_8&=0
\end{split}
\end{equation}

After solving the system (which results in a complex time function for each node), we take the imaginary part of $v_{6f}$ and plot the corresponding sinusoidal function. While normally we would take the real part (a cosine), in this particular case, as the voltage source is a sine function, we must take the imaginary part instead. The values of the complex amplitudes of each node were also determined and are printed in table \ref{tab4}.

\begin{figure}[!htp] \centering
\includegraphics[width=0.6\textwidth]{forced.eps}
\caption{Forced solution $v_{6f}(t)$ in the interval [0,20]ms}
\label{fig3}
\end{figure}
\FloatBarrier

\begin{table}[h]                             
\centering                                  % para centr. tabela
\def\arraystretch{1.2}                       % esp. entre linhas
\begin{tabular}{c|c}                    % lr = 2 col (esq/dir)
\hline                                  % linha topo

\textbf{Name}  & \textbf{Value [V]}\\     

\hline                                % linha
\input{Nodal4_tab}
\hline                                % linha final
\end{tabular}   

\caption{Complex amplitude of each node voltage.}
\label{tab4}   
\end{table}                             
\FloatBarrier

\subsection{Final Total Solution $v_6(t)$}\label{sec2.5}

For $t$ $\in$ [0,20]ms, superimposing the natural and forced responses shown previously, the total solution $v_6$ can be obtained. The result is the sum of a negative exponential, which will approach zero as time passes, and a sinusoidal function, thus the solution will roughly resemble a negative exponential before stabilising as a sine function.  For $t$ $\in$ [-5,0]ms, the voltage source is constant (there is no voltage drop in the capacitor) and as such the graph is a horizontal line. The graph confirming this behaviour is plotted below.

Figure \ref{fig4} represents the total solution $v_6$ in the two aforementioned time intervals, as well as voltage source $v_s$ for comparison.

\begin{figure}[!htp] \centering
\includegraphics[width=0.6\textwidth]{total.eps}
\caption{Total solution $v_6(t)$ and $v_s(t$ in the interval [-5,20]ms}
\label{fig4}
\end{figure}
\FloatBarrier

\subsection{Frequency Responses: $v_s(f)$, $v_c(f)$ and $v_6(f)$}\label{sec2.6}

The final point of this theoretical analysis was to determine the frequency responses of $v_c(f) = v_6(f) - v_8(f)$ and $v_6(f)$. For this purpose, the same system in section \ref{sec2.4} was used, with only the subtle change that $Z_C$ is now a function of frequency, $f$.

After obtaining $v_6(f)$ and $v_8(f)$ it is possible to plot both the amplitude and phase of $v_c(f)$ and $v_6(f)$ and compare them with the voltage source $v_s$, effectively visualising the transfer function.

The transfer function of an RC circuit is $\frac{1}{1+j\omega RC}$. After some mathematical manipulation we derive formulae for both the magnitude and the phase of the output:
\begin{equation}
    V = \sqrt{\frac{1}{1+\omega ^2 R^2C^2}}
\end{equation}
\begin{equation}
    \phi = \arctan(-\omega RC)
\end{equation}

Taking the limit when $\omega$ approaches infinity, we see that the amplitude $V$ approaches 0 while the phase $\phi$ approaches -$\frac{\pi}{2}$.

\begin{figure}[!htp] \centering
\includegraphics[width=0.6\textwidth]{response.eps}
\caption{Amplitude comparison for frequencies in the interval [$10^{-1}$,$10^6$]Hz.}
\label{fig5}
\end{figure}
\FloatBarrier

The complex amplitude is shown in decibel. For the voltage source the amplitude is 0, as could easily be concluded from the fact that its value is $log_{10}(1)$. $v_6(f)$ remains fairly constant throughout the range of frequencies studied, with only a slight change around 100Hz. However, at the same frequencies $v_c(f)$ suddenly plummets and keeps getting smaller at a linear rate. Remembering that the decibel is a logarithmic scale, and that $\lim_{x\to 0} log_{10}(x)$ is $-\infty$, this result can be understood from the amplitude formula derived from the transfer function.

\begin{figure}[!htp] \centering
\includegraphics[width=0.6\textwidth]{phase.eps}
\caption{Phase comparison for frequencies in the interval [$10^{-1}$,$10^6$]Hz.}
\label{fig6}
\end{figure}
\FloatBarrier
The phase is shown in degrees. Once again, as expected, $v_s$ is a horizontal line at 0 degrees. As for $v_c$, the phase difference increases until around 1000Hz where it stays at -90 degrees, like the phase formula predicted for large enough $f$. The voltage in node 6, however, drops to -180 degrees instead of -90 due to the presence of dependent sources.


