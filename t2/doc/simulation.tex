\section{Simulation Analysis Using \emph{NgSpice}}
\label{sec:simulation}

\subsection{Operating point for $t<0$}
\label{sec:3.1}
Table \ref{op:tab_4} shows the simulated operating point results for the circuit under analysis when t$<$0s. It is possible to see that table \ref{op:tab_4} shows the current that goes through all the resistors as well as the voltage in each node. 


However, as we can see there is an additional node when compared to the number of nodes that the circuit truly has. Since the circuit has a current-controlled voltage source, in order to be able to define this component in the \emph{NgSpice} netlist, a current value that goes through another voltage source is required. However, in this circuit, this current value must be the same that goes through the resistor $R_6$. Therefore, an additional voltage source, $V_{aux}$, was added to the circuit between $R_6$ and $R_7$ with voltage equal to 0V, DC. Thus, this additional voltage source will have the same current that goes through $R_6$. A new node 4 that connects $R_6$ to $V_{aux}$, is created. Accordingly, node 7 will connect $V_{aux}$ to $R_7$.\\

\begin{table}[h]
  \centering
  \def\arraystretch{1.2}
 
\begin{tabular}{c|c}
    \hline    
    \textbf{Name} & \textbf{Value [A or V]} \\ \hline
    @c[i] & 0.000000e+00\\ \hline
@gb[i] & 6.290974e-04\\ \hline
@r1[i] & -5.99596e-04\\ \hline
@r2[i] & -6.29097e-04\\ \hline
@r3[i] & 2.950172e-05\\ \hline
@r4[i] & 1.415159e-03\\ \hline
@r5[i] & -1.38566e-03\\ \hline
@r6[i] & 2.014755e-03\\ \hline
@r7[i] & 2.014755e-03\\ \hline
v(1) & 5.225668e+00\\ \hline
v(2) & 5.854771e+00\\ \hline
v(3) & 7.120022e+00\\ \hline
v(4) & -4.20841e+00\\ \hline
v(5) & 5.764941e+00\\ \hline
v(6) & 9.988837e+00\\ \hline
v(7) & -4.20841e+00\\ \hline
v(8) & -6.25008e+00\\ \hline

    \hline
  \end{tabular}
  \captionsetup{justification=centering, margin=2cm}
  \caption{Operating point for t$<$0. Values obtained from the \emph{NgSpice} script. A variable preceded by @ is of type {\em current} and expressed in Ampere; other variables are of type {\it voltage} and expressed in Volt.}
  \label{op:tab_4}
\end{table}
\FloatBarrier

The \emph{Ngspice} netlist is done according to the following "positive" and "negative" nodes associated to the passive elements:
\begin{figure}[!htp] \centering
\includegraphics[width=0.5\textwidth]{t2NgSpice.pdf}
\captionsetup{justification=centering, margin=2cm}
\caption{T2 Circuit with "positive" and "negative" nodes associated to the passive elements}
\label{fig:t2circuitNgSpice}
\end{figure}
\FloatBarrier

\subsection{Operating point for $V_s=0$}
\label{sec:3.2}

Table \ref{op:tab_5} shows the simulated operating point results for the circuit under analysis when $V_s=0$. Furthermore, the capacitor was replaced with an independent voltage source $V_x=V(6)-V(8)$, where $V(6)$ and $V(8)$ are the voltages of the nodes obtained in \ref{sec:3.1}. This change was due to the same reason as explained in %fazer referencia da secção da analise teorica%
The values obtained 

\begin{table}[h]
  \centering
  \def\arraystretch{1.2}
 
\begin{tabular}{c|c}
    \hline    
    \textbf{Name} & \textbf{Value [A or V]} \\ \hline
    @gb[i] & 0.000000e+00\\ \hline
@r1[i] & 0.000000e+00\\ \hline
@r2[i] & 0.000000e+00\\ \hline
@r3[i] & 0.000000e+00\\ \hline
@r4[i] & 0.000000e+00\\ \hline
@r5[i] & -2.72282e-03\\ \hline
@r6[i] & 0.000000e+00\\ \hline
@r7[i] & 0.000000e+00\\ \hline
v(1) & 0.000000e+00\\ \hline
v(2) & 0.000000e+00\\ \hline
v(3) & 0.000000e+00\\ \hline
v(4) & 0.000000e+00\\ \hline
v(5) & 0.000000e+00\\ \hline
v(6) & 8.573530e+00\\ \hline
v(7) & 0.000000e+00\\ \hline
v(8) & 0.000000e+00\\ \hline
Req & 3.148774e+03\\ \hline
Vx & 8.573530e+00\\ \hline
Ix & -2.72282e-03\\ \hline

    \hline
  \end{tabular}
  \captionsetup{justification=centering, margin=2cm}
  \caption{Operating point for $V_s=0$. Values obtained from the \emph{NgSpice} script. A variable preceded by @ is of type {\em current} and expressed in Ampere; other variables are of type {\it voltage} and expressed in Volt.}
  \label{op:tab_5}
\end{table}
\FloatBarrier

\subsection{Natural response of the circuit}
\label{sec:3.3}

In this section the natural response of the circuit is obtained and the result is presented in \ref{transv6}. 

\begin{figure}[!htp] \centering
\includegraphics[width=0.30\textwidth]{transv6.pdf}
\captionsetup{justification=centering, margin=2cm}
\caption{Natural response of the circuit. Voltage of node v(6) (V) in respect to time.}
\label{transv6}
\end{figure}
\FloatBarrier


