\section{Introduction}
\label{sec:introduction}

% state the learning objective 
The objective of this laboratory assignment is to study a circuit containing resistors connected both in parallel and in series to other active components. These resistors are connected to an independent voltage source and dependent voltage-controlled current and current-controlled voltage sources.


In Section \ref{sec:analysis}, a theoretical analysis is presented. The circuit is analysed for $t<0$ with a simple nodal method and for $t>0$ combining the natural response and the forced response, which are determined also using the nodal method. The results are obtained using \emph{Octave} \cite{bib:octave}. In Section \ref{sec:simulation}, the circuit is analysed by simulation using \emph{NgSpice} \cite{bib:ngspice}. The results are then compared to the theoretical results obtained in Section \ref{sec:analysis}. The conclusions of this study are outlined in Section \ref{sec:conclusion}. Below, Figure \ref{fig:t2circuit} shows the circuit that was analysed and the respective node numbering that was chosen in the lab assignment.

\begin{figure}[!htp] \centering
\includegraphics[width=0.7\textwidth]{t2.pdf}
\caption{Circuit T2}
\label{fig:t2circuit}
\end{figure}
\FloatBarrier

In order to start analysing the circuit, some data was needed: the values of the resistances, the capacitance $C$ and $K_b$, $K_d$ and $V_s$, which were all generated by the \emph{Python} script previously handed to us. The following data, using the lowest student number in our group, 95807, was obtained:

\begin{table}[h]                             

\centering                                  % para centr. tabela
                        
\def\arraystretch{1.2}                       % esp. entre linhas
\begin{tabular}{c|c}                    % lr = 2 col (esq/dir)
\hline                                  % linha topo

\textbf{Name}  & \textbf{Value [k$\Omega$, V, A, $\mu$F or mS]}\\     

\hline                                % linha
\input{data_tab}
\hline                                % linha final
\end{tabular}   

\caption{Values for the constants used in the circuit analysis}
\label{tab1}   
\end{table}                             
\FloatBarrier
