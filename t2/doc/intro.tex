\section{Introduction}
\label{sec:introduction}

% state the learning objective 
The objective of this laboratory assignment is to study a circuit containing resistors connected both in parallel and in series to other active components. These resistors are connected to independent voltage and current sources and dependent voltage-controlled current and current-controlled voltage sources.


In Section \ref{sec:analysis}, a theoretical analysis using both the nodal and the mesh method is presented. The results are obtained using \emph{Octave} \cite{bib:octave}. In Section \ref{sec:simulation}, the circuit is analysed by simulation using \emph{NgSpice} \cite{bib:ngspice}. The results are then compared to the theoretical results obtained in Section \ref{sec:analysis}. The conclusions of this study are outlined in Section \ref{sec:conclusion}. Below, Figure \ref{fig:t1circuit} shows the circuit that was analysed and the respective node numbering that the group chose.

\begin{figure}[!htp] \centering
\includegraphics[width=0.7\textwidth]{T1_Circuit.pdf}
\caption{Circuit T1 and respective nodes}
\label{fig:t1circuit}
\end{figure}
\FloatBarrier

In order to start analysing the circuit, some data was needed. This data, such as the values of the resistances, the voltage of $V_a$, the current of $I_d$ and $K_b$ and $K_c$ values, were all generated by the \emph{Python} script, previously handed to us. The following data, using the lowest student number in our group, 95807, was obtained:

\begin{table}[h]                             

\centering                                  % para centr. tabela

\label{tab_1}                           
\def\arraystretch{1.1}                       % esp. entre linhas


\begin{tabular}{c|lr}                    % lr = 2 col (esq/dir)
\hline                                  % linha topo

\textbf{Name}  & \textbf{Value [$\Omega$, V, A or S]}\\     

\hline                                % linha
$R_1$          & 1.04921233729$\times \num{e3}$ \\ 
$R_2$          & 2.01121557182$\times \num{e3}$ \\ 
$R_3$          & 3.04491334831$\times \num{e3}$ \\ 
$R_4$          & 4.07370420497$\times \num{e3}$ \\
$R_5$          & 3.04829678473$\times \num{e3}$ \\
$R_6$          & 2.08879558009$\times \num{e3}$  \\ 
$R_7$          & 1.01335761883$\times \num{e3}$  \\
$V_a$          & 5.22566789497               \\
$I_d$          & 1.04037874222$\times \num{e-3}$ \\
$K_b$          & 7.00318569445$\times \num{e-3}$ \\ 
$K_c$          & 8.05999483103$\times \num{e3}$ \\
\hline                                % linha final
\end{tabular}   
\caption{Values for the constants used in the circuit analysis}
\end{table}  
\FloatBarrier
