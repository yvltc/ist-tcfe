\section{Comparison}
\label{sec:comparison}

\begin{table}
\parbox{.45\linewidth}{
\centering
                        

\centering                                  % para centr. tabela
                        
\def\arraystretch{1}                       % esp. entre linhas

\begin{tabular}{c|c}                    % lr = 2 col (esq/dir)
\hline                                  % linha topo

\textbf{Name}  & \textbf{Value [A or V]}\\     

\hline                                % linha
\input{Nodal1_tab}
\hline                                % linha final
\end{tabular}

\caption{Table \ref{tab2}}
\label{tab7}
}
\hfill
\parbox{.45\linewidth}{
\centering

  \centering
  \def\arraystretch{1}
 
\begin{tabular}{c|c}
    \hline    
    \textbf{Name} & \textbf{Value [A or V]} \\ \hline
    @c[i] & 0.000000e+00\\ \hline
@gb[i] & 6.290974e-04\\ \hline
@r1[i] & -5.99596e-04\\ \hline
@r2[i] & -6.29097e-04\\ \hline
@r3[i] & 2.950172e-05\\ \hline
@r4[i] & 1.415159e-03\\ \hline
@r5[i] & -1.38566e-03\\ \hline
@r6[i] & 2.014755e-03\\ \hline
@r7[i] & 2.014755e-03\\ \hline
v(1) & 5.225668e+00\\ \hline
v(2) & 5.854771e+00\\ \hline
v(3) & 7.120022e+00\\ \hline
v(4) & -4.20841e+00\\ \hline
v(5) & 5.764941e+00\\ \hline
v(6) & 9.988837e+00\\ \hline
v(7) & -4.20841e+00\\ \hline
v(8) & -6.25008e+00\\ \hline

    \hline
  \end{tabular}
  \captionsetup{justification=centering, margin=2cm}
  \caption{Table \ref{op:tab_4}}
  \label{tab8}
}
\end{table}
\FloatBarrier


\begin{table}
\parbox{.45\linewidth}{
\centering
                        

\centering                                  % para centr. tabela
                        
\def\arraystretch{1}                       % esp. entre linhas
\begin{tabular}{c|c}                    % lr = 2 col (esq/dir)
\hline                                  % linha topo

\textbf{Name}  & \textbf{Value [A, V, $\Omega$ or s]}\\     

\hline                                % linha
\input{Nodal2_tab}
\hline                                % linha final
\end{tabular}

\caption{Table \ref{tab3}}
\label{tab9}
}
\hfill
\parbox{.45\linewidth}{
\centering

  \centering
  \def\arraystretch{1}
 
\begin{tabular}{c|c}
    \hline    
    \textbf{Name} & \textbf{Value [A, V or $\Omega$]} \\ \hline
    @gb[i] & 0.000000e+00\\ \hline
@r1[i] & 0.000000e+00\\ \hline
@r2[i] & 0.000000e+00\\ \hline
@r3[i] & 0.000000e+00\\ \hline
@r4[i] & 0.000000e+00\\ \hline
@r5[i] & -2.72282e-03\\ \hline
@r6[i] & 0.000000e+00\\ \hline
@r7[i] & 0.000000e+00\\ \hline
v(1) & 0.000000e+00\\ \hline
v(2) & 0.000000e+00\\ \hline
v(3) & 0.000000e+00\\ \hline
v(4) & 0.000000e+00\\ \hline
v(5) & 0.000000e+00\\ \hline
v(6) & 8.573530e+00\\ \hline
v(7) & 0.000000e+00\\ \hline
v(8) & 0.000000e+00\\ \hline
Req & 3.148774e+03\\ \hline
Vx & 8.573530e+00\\ \hline
Ix & -2.72282e-03\\ \hline

    \hline
  \end{tabular}
  \captionsetup{justification=centering, margin=2cm}
  \caption{Table \ref{op:tab_5}}
  \label{tab10}
}
\end{table}
\FloatBarrier

Looking at the results shown in table \ref{op:tab_4} and \ref{op:tab_5} and comparing them to the theoretical analysis present in tables \ref{tab2} and \ref{tab3}, it is clear that the \emph{NgSpice} simulation results were equal to the values predicted by the methods applied.

Looking at tables \ref{tab2} and \ref{op:tab_4}, the first thing to note is that $v(4)$ and $v(7)$ are the same theoretical node (like the equal values confirm) but had to be both be included due to the way \emph{NgSpice} works. Thus, ignoring $v(4)$, every simulated value has an exact correspondence in the theoretical model.

As for tables \ref{tab3} and \ref{op:tab_5}, once again the results match. Every node expect node 6 has zero voltage in both the simulation and the theoretical analysis, while every current except the current flowing through resistor $R_5$ and the capacitor is zero. Note that the values for the capacitor current are symmetrical due to the orientation defined in the theoretical and in the simulation being symmetrical.

\section{Conclusion}
\label{sec:conclusion}
 As the "experimental" results are merely a simulation based on the same theoretical models used in this report, it is not surprising that this is the case. The circuit is a first order system (current and voltage sources, resistors and a capacitor) and therefore the manipulated equations were linear and first order linear differential equations. Symbolic calculations were required to compute and solve them in a mathematical software like \emph{Octave}. 
