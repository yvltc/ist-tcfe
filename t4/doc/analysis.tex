\section{Analysis}
\label{sec:analysis}
\subsection{Theoretical explanation}

The circuit is divided in two different stages, a gain stage with a NPN transistor and an output stage with a PNP transistor. Thus, for the gain stage's operating point (OP), the voltage drop between the collecter and the emitter, $V_{CE}$, will have to be higher than the voltage drop between the base and the emitter, $V_{BEON}$, to confirm that it operates in the forward active region (FAR). Similarly, in the output stage, $V_{EC}$ will have to be bigger than $V_{EBON}$, since this is a PNP transistor. For a silicon transistor, this voltage drop is approximately 0.7 V.

For the NPN transistor (the equation is similar for the PNP transistor, but with emitter-collecter and emitter-base instead):
\begin{equation}
    V_{CE} = V_O - V_E > V_{BEON}
\end{equation}
where $V_O$ (output voltage of the gain stage) and $V_E$ (voltage in the emitter node) can be determined through mesh analysis as seen in the theoretical classes.

In order to ensure FAR, a bias circuit composed of two resistors $R_1$ and $R_2$ and a supply voltage $V_{CC}$ needs to be added to the circuit, like seen in the circuit drawing, because the input voltage has 0V DC value and so cannot forwardly bias the junction. A coupling capacitor $C_I$ is also added to block the 0V DC component of the input voltage. Similarly, another coupling capacitor $C_O$ needs to be added between the output stage and load. 

The main goal of the gain stage, as the name indicates, is to get a high voltage gain and a high input impedance so as to not degrade the input signal.  A third capacitor needs to be added to the gain stage (in parallel with resistor $R_E$) to compensate the disadvantage of this resistor: while it stabilises temperature effect, it lowers the gain. The bypass capacitor $C_B$ acts as an open-circuit for low frequencies (DC) and a short-circuit for medium to high frequencies (AC). In DC, temperature stabilisation is more important; in AC, gain is more important. This capacitor then solves the problem of the lower gain caused by the resistor. However, this stage comes with a cost. The output impedance of the gain stage is high, which is incompatible with the load (speaker).

The output stage fixes the high output impedance of the gain stage by having a high input impedance and a low output impedance compatible with the load. Since the voltage has already been amplified, the gain of the output stage should, ideally, be 1.

\begin{table}[H]
\centering
    \begin{tabular}{c|c}        % lr = 2 col (esq/dir)
\hline                      % linha topo

\textbf{Name}  & \textbf{Value}\\     
\hline                      % linha
\input{gain_stage_tab}
\hline                      % linha final
\end{tabular}
    \caption{Gain Stage.}
 \label{tab_gain}
\end{table} 



The gain stage formula used to compute the value of the voltage gain in the gain stage is:
\begin{equation}
    G_1=\frac{v_o}{v_i}= \frac{R_1//R_2}{R_{in}+R_1//R_2}R_C\frac{R_E-g_m r_\pi r_o}{(r_o+R_C+R_E)((R_1//R_2)//R_{in}+r_\pi+R_E)+g_m R_E r_\pi r_o-R_E^2}
\end{equation}

The values of impedance presented in table \ref{tab_gain} were calculated with

\begin{equation}
    Z1_I=\frac{1}{\frac{1}{R_B}+\frac{1}{\frac{(r_{o}+R_{C}+R_{E})(r_{\pi}+R_{E})+g_{m} R_{E} r_{o} r_{\pi} - R_{E}^2}{r_{o}+R_{C}+R_{E}}}}
\end{equation}
\begin{equation}
    Z1_O=\frac{1}{\frac{1}{r_{o1}}+\frac{1}{R_{C1}}}
\end{equation}


\begin{table}[H]
\centering
    \begin{tabular}{c|c}        % lr = 2 col (esq/dir)
\hline                      % linha topo

\textbf{Name}  & \textbf{Value}\\     
\hline                      % linha
\input{output_stage_tab}
\hline                      % linha final
\end{tabular}
    \caption{Output Stage.}
 \label{tab_output}
\end{table} 
The output gain formula used to compute these values is given by:
\begin{equation}
    G_1=\frac{v_o}{v_i}=\frac{g_m}{g_m+g_E+g_o+g_\pi}
\end{equation}
where $g_m$, $g_\pi$, $g_o$ and $g_E$ are conductances.

The values of impedance presented in table \ref{tab_output} were calculated with
\begin{equation}
    Z2_I=\frac{g_{\pi}+g_{E}+g_{o}+g_{m}}{g_{\pi}(g_{\pi}+g_{E}+g_{o})}
\end{equation}
\begin{equation}
    Z2_O=\frac{1}{g_{\pi}+g_{E}+g_{o}+g_{m}}
\end{equation}

Note that $r_o$, $r_\pi$ and $r_m$ and their equivalent conductances are transistor resistors.

\subsection{Comparing the theoretical and the simulated results}

\begin{table}[H]
\parbox{.5\linewidth}{
\centering                
\def\arraystretch{1}        % esp. entre linhas

\begin{tabular}{c|c}        % lr = 2 col (esq/dir)
\hline                      % linha topo

\textbf{Name}  & \textbf{Value}\\     
\hline                      % linha
@c[i] & 0.000000e+00\\ \hline
@gb[i] & 6.290974e-04\\ \hline
@r1[i] & -5.99596e-04\\ \hline
@r2[i] & -6.29097e-04\\ \hline
@r3[i] & 2.950172e-05\\ \hline
@r4[i] & 1.415159e-03\\ \hline
@r5[i] & -1.38566e-03\\ \hline
@r6[i] & 2.014755e-03\\ \hline
@r7[i] & 2.014755e-03\\ \hline
v(1) & 5.225668e+00\\ \hline
v(2) & 5.854771e+00\\ \hline
v(3) & 7.120022e+00\\ \hline
v(4) & -4.20841e+00\\ \hline
v(5) & 5.764941e+00\\ \hline
v(6) & 9.988837e+00\\ \hline
v(7) & -4.20841e+00\\ \hline
v(8) & -6.25008e+00\\ \hline

\hline                      % linha final
\end{tabular}
\captionsetup{justification=justified, margin=0.5cm} 
\caption{Theoretical Operating Point: voltages in V.}
\label{tab7}
}
\hfill
\parbox{.5\linewidth}{
\centering
\def\arraystretch{1}

\begin{tabular}{c|c}
\hline    
\textbf{Name} & \textbf{Value} \\ \hline
\input{OPngspice_tab}
\hline
\end{tabular}
\captionsetup{justification=justified, margin=0.5cm} 
\caption{Simulated Operating Point: voltages in V.}
\label{tab8}
}
\end{table}

Right from the beginning it is clear that the values of $V_{O1}$, $V_{CE}$, $V_{EC}$ and $V_{O2}$ are lower than the simulated results. The other results for the variables $V_{BEON}$/$v_{be}$ and $V_{EBON}$/$v_{eb}$ differ a few decimals between the simulated and theoretical results. The NgSpice \cite{bib:ngspice} models from which the entire NgSpice simulation was made are more complex and more precise than the calculations made in Octave \cite{bib:octave}, given that were made several approximations along the entire process.
\begin{table}[H]
\parbox{.5\linewidth}{
\centering                
\def\arraystretch{1}        % esp. entre linhas

\begin{tabular}{c|c}        % lr = 2 col (esq/dir)
\hline                      % linha topo

\textbf{Name}  & \textbf{Value}\\     
\hline                      % linha
\input{total_tab}
\hline                      % linha final
\end{tabular}
\captionsetup{justification=justified, margin=0.5cm} 
\caption{Theoretical results. Impedances in $\Omega$, cost in MU, frequencies in Hz.}
\label{tab9}
}
\hfill
\parbox{.5\linewidth}{
\centering
\def\arraystretch{1}

\begin{tabular}{c|c}
\hline    
\textbf{Name} & \textbf{Value} \\ \hline
\input{data_tab}
\hline
\end{tabular}
\captionsetup{justification=justified, margin=0.5cm} 
\caption{Simulation results. Impedances in $\Omega$, cost in MU, frequencies in Hz.}
\label{tab10}
}
\end{table}

Note that the upper cutoff frequency obtained in the Octave table is the same as the NgSpice value because it was directly obtained from the simulation. The lower cutoff frequency, however, was not taken from the simulation and thus differs.

In order to simulate the output impedance of the circuit, a different circuit was used as it is shown in \ref{z_out}
 
\begin{figure}[H] \centering
\includegraphics[width=0.7\textwidth]{t4_zout.pdf}
\caption{Circuit used to calculate the output impedance}
\label{z_out}
\end{figure}


Furthermore, the following plots were obtained from \emph{Octave} and \emph{NgSpice}:
\begin{figure}[H]
\centering
\begin{subfigure}{0.5\textwidth}
  \centering
  \includegraphics[width=1.1\linewidth]{gain_dB.eps}
  \caption{Octave plot}
  \label{fig:sub1}
\end{subfigure}%
\begin{subfigure}{.5\textwidth}
  \centering
  \includegraphics[width=0.9\linewidth]{vo2f.eps}
  \caption{NgSpice plot}
  \label{fig:sub2}
\end{subfigure}
\caption{$V_{gain}$ as a function of $f$ (dB)}
\label{fig:test}
\end{figure}

It is possible to see that both plots are very similar, with the exception that the voltage on the \emph{NgSpice} plot starts dropping at a higher frequency and on \emph{Octave} plot it does not drop. This happens because in \emph{Octave} the Upper Cutoff Frequency is not considered, given that it is dependent on the transistors and its analysis is complex procedure, whereas in \emph{NgSpice} it is considered. Therefore, when the Upper Cutoff Frequency is reached the voltage should start dropping.

